%%
%% Copyright (c) 2018-2019 Weitian LI <wt@liwt.net>
%% CC BY 4.0 License
%%
%% Created: 2018-04-11
%%

% Chinese version
\documentclass[zh]{resume}

% Adjust icon size (default: same size as the text)
\iconsize{\Large}

% File information shown at the footer of the last page
\fileinfo{%
  \faCopyright{} 2021, Yulong LI \hspace{0.5em}
  \creativecommons{by}{1.0} \hspace{0.5em}
  \githublink{thylakoids}{resume} \hspace{0.5em}
  \faEdit{} \today
}

\name{玉龙}{李}

\keywords{BSD, Linux, Programming, Python, C, Shell, DevOps, SysAdmin}

% \tagline{\icon{\faBinoculars}} <position-to-look-for>}
% \tagline{<current-position>}

% \photo{<height>}{<filename>}

\profile{
  \mobile{182-1739-2001}
  \email{liyulongsjtu@hotmail.com}
  \github{thylakoids} \\
  \university{上海交通大学}
  \degree{生命科学 \textbullet 学士}
  \birthday{1993-06-03}
  \address{上海}
  % Custom information:
  % \icontext{<icon>}{<text>}
  % \iconlink{<icon>}{<link>}{<text>}
}

\begin{document}
\makeheader

%======================================================================
% Summary & Objectives
%======================================================================
{\onehalfspacing\hspace{2em}%
有生物, 医学, 计算机等复合背景, 擅长数据建模分析, 数据可视化, 热衷计算机和网络
技术. 了解大多数医疗器械的基本原理, 熟悉医疗影像图像格式.有多年linux使用经验,
vim忠实用户, 熟练掌握Python, Matlab, JavaScript等 语言. 对机器学习, 机器视觉,
JS逆向等有浓厚的兴趣.
\par}

%======================================================================
\sectionTitle{技能和语言}{\faWrench}
%======================================================================
\begin{competences}
  \comptence{操作系统}{%
    \icon{\faLinux} Linux (6 年)
  }
  \comptence{编程}{%
    Python, Matlab, JavaScript, Shell, R, C
  }
  \comptence{工具}{%
    Vim, Tmux, SSH, Git, Markdown, \LaTeX
  }
  \comptence{数据分析}{%
    Numpy,Pandas, R; Matplotlib, ggplot2; PyTorch, Scikit-learn
  }
  \comptence{网站开发}{%
    \icon{\faReact}React, JavaScript, Flask
  }
  \comptence{语言}{
    \textbf{英语} --- 听读(优良),写说(日常交流)
  }
\end{competences}

%======================================================================
\sectionTitle{教育背景}{\faGraduationCap}
%======================================================================
\begin{educations}
  \education%
    {2015.09}%
    [2019.09]%
    {上海交通大学}%
    {生物医学工程学院}%
    {生物医学工程}%
    {博士(结业)}

  \separator{0.5ex}
  \education%
    {2011.09}%
    [2015.06]%
    {上海交通大学}%
    {致远学院}%
    {生命科学}%
    {学士}
\end{educations}


%======================================================================
\sectionTitle{实习经历}{\faBriefcase}
%======================================================================
\begin{experiences}
  \experience%
    [2018.10]%
    {2019.05}%
    {全栈开发 @ 志御(初创公司)}%
    [\begin{itemize}
      \item{前端 (Python, PyQt, VTK): 实现医学图像(Dicom)的可视化与交互展示}
      \item{算法 (CNN, Keras, OpenCV):  使用深度学习算法实现病灶的分割和分型}
    \end{itemize}]

  \separator{0.5ex}
  \experience%
    [2018.01]%
    {2018.06}%
    { 研究助理 @ 飞利浦研究院}%
    [\begin{itemize}
      \item {使用U-net独立完成肺结节检测(Luna16数据集)}
      \item {集成Flask+Angular项目, 实现医学图像自动分析, 并展示Lirads页面}
    \end{itemize}]
\end{experiences}


%======================================================================
\sectionTitle{工作经历}{\faLaptopCode}
%======================================================================
\begin{experiences}
  \experience%
    [2020.11]%
    {2021.06}%
    {全栈开发 @ 志御(初创公司)}%
    [\begin{itemize}
      \item{前端 (React, JavaScript, CSS): 负责Web前端功能设计与开发, 完成医学图像交互动态展示}
      \item{医学图像 (Dicom, NIFTI): 深入了解医学图像存储和传输协议}
      \item{其他: 负责算法优化, 开发流程规范化, 技术文档撰写, 任务分配等工作}
    \end{itemize}]
\end{experiences}


%======================================================================
\sectionTitle{个人项目}{\faCode}
%======================================================================
\begin{itemize}
  \item \link{https://github.com/thylakoids/vimrc}{\texttt{vimrc}}:
    (vim script)
    个人vim配置文件, 也包含了开发环境搭建, 技术知识总结(vimwiki)
  \item \link{https://github.com/thylakoids/donut-in-terminal}{\texttt{donut-in-terminal}}:
    (computer vision, linear algebra)
    在终端用字符渲染一个旋转的甜甜圈
  \item \link{https://github.com/thylakoids/esdeobfuscate}{\texttt{esdeobfuscate}}:
    (JavaScript, AST)
    用AST技术部分执行JS程序
  \item \link{https://github.com/thylakoids/luna}{\texttt{luna}}:
    (Keras, UNet)
    深度学习检测肺结节
  \item \link{https://github.com/thylakoids/CitationNet}{\texttt{CitationNet}}:
    (Python, JavaScript)
    用网络的方式展示PubMed文献之间的引用信息
  \item \link{https://github.com/thylakoids/resume}{\texttt{resume}}:
    (\LaTeX)
    此简历的模板和源文件
\end{itemize}


\end{document}
